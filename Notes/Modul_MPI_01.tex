\documentclass[bahasa,a4paper,12pt]{extarticle}
\usepackage[a4paper]{geometry}
\geometry{verbose,tmargin=1.5cm,bmargin=1.5cm,lmargin=2.5cm,rmargin=2.5cm}

\setlength{\parskip}{\smallskipamount}
\setlength{\parindent}{0pt}

\usepackage{mathptmx}

%\usepackage{fontspec}
%\defaultfontfeatures{Ligatures=TeX}
%\setmainfont{Linux Libertine O}

\usepackage{hyperref}
\usepackage{url}
\usepackage{xcolor}
\usepackage{babel}

\usepackage{minted}
\newminted{fortran}{breaklines,fontsize=\footnotesize}
\newminted{text}{breaklines,fontsize=\footnotesize}
\newminted{C}{breaklines,fontsize=\footnotesize}

\definecolor{mintedbg}{rgb}{0.95,0.95,0.95}
\usepackage{mdframed}

\BeforeBeginEnvironment{minted}{\begin{mdframed}[backgroundcolor=mintedbg]}
\AfterEndEnvironment{minted}{\end{mdframed}}

\begin{document}

\title{%
  \large Pemrograman Paralel dengan Message Passing Interface (MPI) \\
  \Large Modul 1: Pengenalan MPI}
\author{Fadjar Fathurrahman}
\date{}
\maketitle

\section{Tujuan}
\begin{enumerate}
\item mengenal MPI and berbagai macam implementasinya
\item mampu membuat program paralel sederhana dengan MPI
\end{enumerate}

\section{Perangkat lunak yang diperlukan}
\begin{itemize}
\item Komputer multicore atau kluster.
\item Kompiler MPI (Open MPI)
\end{itemize}

\section{Pendahuluan}

Apakah \textit{message passing interface} tersebut?

Perbandingan dengan paradigma pemrograman paralel yang lain.


\section{Program sederhana}

Mari kita mulai dengan program yang sederhana:
\inputminted[breaklines,fontsize=\footnotesize]{fortran}{codes/f90/01_simple.f90}

Perhatikan bahwa:
\begin{fortrancode}
IMPLICIT NONE
USE 'mpi.h'
\end{fortrancode}

Cara yang lebih modern, yaitu dengan menggunakan modul.
\begin{fortrancode}
USE mpi
IMPLICIT NONE
\end{fortrancode}

Bangun program dengan menggunakan perintah:
\begin{textcode}
mpifort 01_simple.f90 -o 01_simple.x
\end{textcode}

Program dapat dijalankan dengan menggunakan perintah
\begin{textcode}
mpirun -np ./01_simple.x
\end{textcode}

Contoh keluaran program (menggunakan 2 proses):
\begin{textcode}
efefer:f90$ mpirun -np 2 ./01_simple.x 
 Hello, I am proc:            1  from            2  total processes
 Hello, I am proc:            0  from            2  total processes
\end{textcode}

Setiap program MPI harus diawali dengan pemanggilan fungsi
\verb|MPI_Init()| dan diakhiri dengan \verb|MPI_Finalize()|.
\begin{fortrancode}
CALL MPI_Init(ierr)
! ... kode perhitungan di sini
CALL MPI_Finalize(ierr)
\end{fortrancode}
Dalam Fortran, dua fungsi tersebut memerlukan satu argumen yaitu \verb|ierr|
yang bertipe \verb|INTEGER|.

Dua fungsi lain yang akan sering kita gunakan adalah \verb|MPI_Comm_size()|
dan \verb|MPI_Comm_rank()|.
\begin{fortrancode}
CALL MPI_Comm_size(MPI_COMM_WORLD, Nprocs, ierr)
CALL MPI_Comm_rank(MPI_COMM_WORLD, rank, ierr)
\end{fortrancode}

\verb|MPI_Comm_size()| digunakan untuk memperoleh informasi mengenai jumlah
proses yang digunakan pada suatu komunikator MPI.

\verb|MPI_Comm_rank()| digunakan untuk memperoleh informasi mengenai rank
dari proses yang sedang berjalan pada suatu komunikator MPI.

Apa yang terjadi jika kita memesan jumlah prosesor yang lebih banyak
daripada jumlah prosesor yang ada pada sistem kita?
\begin{textcode}
efefer:f90$ mpirun -np 3 ./01_simple.x 
--------------------------------------------------------------------------
There are not enough slots available in the system to satisfy the 3 slots
that were requested by the application:
  ./01_simple.x

Either request fewer slots for your application, or make more slots available
for use.
--------------------------------------------------------------------------
\end{textcode}



\end{document}


