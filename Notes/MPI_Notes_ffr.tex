\documentclass[bahasa,a4paper,12pt]{extarticle}
\usepackage[a4paper]{geometry}
\geometry{verbose,tmargin=1.5cm,bmargin=1.5cm,lmargin=1cm,rmargin=1cm}

\setlength{\parskip}{\smallskipamount}
\setlength{\parindent}{0pt}

\usepackage{fontspec}
\defaultfontfeatures{Ligatures=TeX}
\setmainfont{Linux Libertine O}

\usepackage{hyperref}
\usepackage{url}
\usepackage{xcolor}
\usepackage{babel}

\usepackage{minted}
\newminted{fortran}{breaklines,fontsize=\footnotesize,texcomments=true}
\newminted{C}{breaklines,fontsize=\footnotesize}

\definecolor{mintedbg}{rgb}{0.95,0.95,0.95}
\usepackage{mdframed}


\begin{document}


\title{Pengenalan Message Passing Interface (MPI)}
\author{Fadjar Fathurrahman}
\maketitle

\section{Pendahuluan}

Apakah \textit{message passing interface} tersebut?


\section{Program sederhana}

Mari kita mulai dengan program yang sederhana:
\begin{fortrancode}
PROGRAM simple01
  IMPLICIT NONE
END PROGRAM
\end{fortrancode}


\end{document}


